
\documentclass[12pt]{article}
\usepackage{amsmath}
\usepackage{graphicx,epsfig, wrapfig}
\usepackage{amssymb}
\usepackage{setspace}
\usepackage{amsthm}
\usepackage{subfig}
\usepackage{amsmath}
%\usepackage{chicago}
\usepackage{natbib}
\usepackage{enumerate}
\usepackage{url}
\usepackage{epstopdf}
\usepackage{fancyhdr}
\usepackage{threeparttable}
\usepackage{caption}
\usepackage{booktabs}
\usepackage{float}
\usepackage{array}
\usepackage{multirow}
%\usepackage[T1]{fontenc}
%\usepackage[utf8]{inputenc}
%\usepackage{authblk}
\newcommand{\keyword}[1]{\textsf{\slshape #1}}

 
\setcounter{MaxMatrixCols}{10}    

%\setstretch{3}
\onehalfspacing
%\setlength{\footnotesep}{0.5cm}
%\doublespacing
\newtheorem{theorem}{Theorem}   
\newtheorem{acknowledgment}[theorem]{Acknowledgment}   
\newtheorem{algorithm}[theorem]{Algorithm}
\newtheorem{axiom}[theorem]{Axiom}
\newtheorem{case}[theorem]{Case} 
\newtheorem{claim}[theorem]{Claim} 
\newtheorem{conclusion}[theorem]{Conclusion}
\newtheorem{condition}[theorem]{Condition}
\newtheorem{conjecture}[theorem]{Conjecture}
\newtheorem{corollary}[theorem]{Corollary}
\newtheorem{criterion}[theorem]{Criterion}
\newtheorem{definition}[theorem]{Definition}
\newtheorem{example}[theorem]{Example}
\newtheorem{exercise}[theorem]{Exercise}
\newtheorem{lemma}{Lemma}
\newtheorem{notation}[theorem]{Notation}
\newtheorem{problem}[theorem]{Problem}
\newtheorem{ques}[theorem]{Question}
\newtheorem{proposition}{Proposition}
\newtheorem{remark}[theorem]{Remark}
\newtheorem{solution}[theorem]{Solution}
\newtheorem{summary}[theorem]{Summary} 
\setlength{\oddsidemargin}{0mm}
\setlength{\textwidth}{6.5in}
\setlength{\topmargin}{0in}
\setlength{\headheight}{0in}
\setlength{\headsep}{0in}
\setlength{\textheight}{9in}


\newtheorem{prop}{Proposition}
 

\makeatletter
\newcommand*{\rom}[1]{\expandafter\@slowromancap\romannumeral #1@}
\makeatother

\newcounter{saveenumi}
\newcommand{\seti}{\setcounter{saveenumi}{\value{enumi}}}
\newcommand{\conti}{\setcounter{enumi}{\value{saveenumi}}}

\begin{document}


\setcounter{footnote}{0}
\title{Mathematics for Economics PhD: Assignment 4}
\author{\textsc{Mariam Arzumanyan,}\thanks {   E-mail: mariama2@illinois.edu.} \\ } 


\date{August 19, 2021}


\maketitle 

\begin{ques}
Solve the following minimization problems:
\begin{enumerate}
    \item \[\min_{x,y} p_xx+p_yy\]
    such that \[\alpha\ln x+(1-\alpha)\ln y\geq u.
    \]
    \item \[\min_{x,y} p_xx+p_yy\]
    such that \[ x^\alpha y^{(1-\alpha)}\geq u.
    \]
    \item Do your answers in part 1 and 2 coincide? Explain.
\end{enumerate}
\end{ques}

\newpage
\begin{ques}
Solve the following minimization problems:
\begin{enumerate}
    \item \[\min_{x,y} p_xx+p_yy\]
    such that \[\sqrt{x}+\sqrt{y}\geq u.
    \]
    \[x\geq 0, y\geq 0.
    \]
    \item \[\min_{x,y} p_xx+p_yy\]
    such that \[ x+b\ln y \geq u\]
    for $(x,y)\in (-\infty, \infty)\times(0, \infty)$
\end{enumerate}
\end{ques}


\newpage
\begin{ques}
Solve the following problem:
\[\max_{x,y} min\{x-y, y-x\}
\]
such that \[p_xx+p_yy=w,
\]\[x\geq 0, y\geq 0.
    \]
\end{ques}

\newpage
\begin{ques}
Solve the following problem:
\[\max_{x_1,x_2, x_3, x_4} x_1^a x_2^b
\]
such that \[x_3+2\ln x_4\geq u
\]\[x_1+x_3=4, x_2+x_4=4.
    \]
\end{ques}

\newpage
\begin{ques}
Solve the following problem:
\[\max_{x_1,x_2, x_3, x_4} ax_1+ bx_2
\]
such that \[\max \{ 2x_3, x_4\}\geq u
\]\[x_1+x_3=1,\]
\[x_2+x_4=1.
    \]
\end{ques}
\newpage
\begin{ques}
Solve the following problems:
\[\max_{x_1,x_2, x_3, x_4} x_1^2+ x_2^2
\]
such that \[\min \{ x_3, x_4\}\geq u
\]\[x_1+x_3=2,\]
\[x_2+x_4=2.
    \]
And the next problem:
\[\max_{x_1,x_2, x_3, x_4} \min \{ x_3, x_4\}
\]
such that \[x_1^2+ x_2^2\geq u
\]\[x_1+x_3=2,\]
\[x_2+x_4=2.
    \]
    Are the solutions of these two problems identical? Explain.
\end{ques}

\newpage
\begin{ques}
Solve the following problem:
\[\max_{x_1,x_2, x_3, x_4} \sqrt{x_1}+\sqrt{ x_2}
\]
such that \[x_3+ x_4^2\geq u
\]\[x_1+x_3=1, x_2+x_4=1.
    \]
\end{ques}

\newpage
\begin{ques}
Find the general solution of the following system of difference equations:
\begin{enumerate}
    \item \begin{align*}
        \begin{array}{c}
             x_{n+1}=3x_n\\
             y_{n+1}=x_n+2y_n;
        \end{array}
    \end{align*}
    
     \item \begin{align*}
        \begin{array}{c}
             x_{n+1}=y_n\\
             y_{n+1}=-x_n+5y_n;
        \end{array}
    \end{align*}
     \item \begin{align*}
        \begin{array}{c}
             x_{n+1}=x_n-y_n\\
             y_{n+1}=2x_n+4y_n;
        \end{array}
    \end{align*}
    
     \item \begin{align*}
        \begin{array}{c}
             x_{n+1}=3x_n-y_n\\
             y_{n+1}=-x_n+2y_n-z_n\\
             z_{n+1}=-y_n+3z_n.
        \end{array}
    \end{align*}
      \item \begin{align*}
        \begin{array}{c}
             x_{n+1}=4x_n-2y_n-2z_n\\
             y_{n+1}=y_n\\
             z_{n+1}=x_n+z_n.
        \end{array}
    \end{align*}
\end{enumerate}
\end{ques}

\newpage
\begin{ques}
Find the general solution for the following equations:
 \begin{enumerate}
        \item $y'=ay,$ where $a$ is a constant.
        \item $y'=ay+b$.
        \item $y'=a(t)y$. 
        \item $y'=a(t)y+b(t)$.
        \item The general separable equation: $y'=a(y)h(t)$.
        \item $y'=y^2$. 
        \item $y'=y(a-by)$.
    \end{enumerate}
\end{ques}
\newpage
\begin{ques}
Solve the following initial value problem:
\begin{enumerate}
    \item $y^{\prime \prime}-y=0$, $y(0)=y'(0)=1.$
    \item $y^{\prime \prime}-5y'+6y=0$, $y(0)=3$, $y'(0)=7.$
\end{enumerate}
\end{ques}

\newpage
\begin{ques}
Find the indefinite integral of each of the following functions:
\begin{enumerate}
    \item $4x^6-x^3$.
    \item $12x^2-6x^{\frac{1}{2}}+3x^{-\frac{1}{2}}-x^{-1}$.
    \item $6e^{7x}$.
    \item $e^{3x^2+6x}(x+1)$.
    \item $(x^2+2x+4)^{\frac{1}{2}}(x+1)$.
    \item $$\frac{3x^{\frac{1}{2}}+x^{-\frac{1}{2}}}{x^{\frac{3}{2}}+x^{\frac{1}{2}}}$$
\end{enumerate}
\end{ques}
\newpage
\begin{ques}
Use integration by parts to integrate:
\begin{enumerate}
    \item \[\int x \ln xdx,
    \]
    \item \[\int x^2 e^{2x} dx.
    \]
\end{enumerate}
\end{ques}
\end{document}
