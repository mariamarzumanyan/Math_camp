
\documentclass[12pt]{article}
\usepackage{amsmath}
\usepackage{graphicx,epsfig, wrapfig}
\usepackage{amssymb}
\usepackage{setspace}
\usepackage{amsthm}
\usepackage{subfig}
\usepackage{amsmath}
%\usepackage{chicago}
\usepackage{natbib}
\usepackage{enumerate}
\usepackage{url}
\usepackage{epstopdf}
\usepackage{fancyhdr}
\usepackage{threeparttable}
\usepackage{caption}
\usepackage{booktabs}
\usepackage{float}
\usepackage{array}
\usepackage{multirow}
%\usepackage[T1]{fontenc}
%\usepackage[utf8]{inputenc}
%\usepackage{authblk}
\newcommand{\keyword}[1]{\textsf{\slshape #1}}

 
\setcounter{MaxMatrixCols}{10}    

%\setstretch{3}
\onehalfspacing
%\setlength{\footnotesep}{0.5cm}
%\doublespacing
\newtheorem{theorem}{Theorem}   
\newtheorem{acknowledgment}[theorem]{Acknowledgment}   
\newtheorem{algorithm}[theorem]{Algorithm}
\newtheorem{axiom}[theorem]{Axiom}
\newtheorem{case}[theorem]{Case} 
\newtheorem{claim}[theorem]{Claim} 
\newtheorem{conclusion}[theorem]{Conclusion}
\newtheorem{condition}[theorem]{Condition}
\newtheorem{conjecture}[theorem]{Conjecture}
\newtheorem{corollary}[theorem]{Corollary}
\newtheorem{criterion}[theorem]{Criterion}
\newtheorem{definition}[theorem]{Definition}
\newtheorem{example}[theorem]{Example}
\newtheorem{exercise}[theorem]{Exercise}
\newtheorem{lemma}{Lemma}
\newtheorem{notation}[theorem]{Notation}
\newtheorem{problem}[theorem]{Problem}
\newtheorem{ques}[theorem]{Question}
\newtheorem{proposition}{Proposition}
\newtheorem{remark}[theorem]{Remark}
\newtheorem{solution}[theorem]{Solution}
\newtheorem{summary}[theorem]{Summary} 
\setlength{\oddsidemargin}{0mm}
\setlength{\textwidth}{6.5in}
\setlength{\topmargin}{0in}
\setlength{\headheight}{0in}
\setlength{\headsep}{0in}
\setlength{\textheight}{9in}


\newtheorem{prop}{Proposition}
 

\makeatletter
\newcommand*{\rom}[1]{\expandafter\@slowromancap\romannumeral #1@}
\makeatother

\newcounter{saveenumi}
\newcommand{\seti}{\setcounter{saveenumi}{\value{enumi}}}
\newcommand{\conti}{\setcounter{enumi}{\value{saveenumi}}}

\begin{document}


\setcounter{footnote}{0}
\title{Mathematics for Economics PhD: Assignment \rom{1}}
\author{\textsc{Mariam Arzumanyan,}\thanks {   E-mail: mariama2@illinois.edu.} \\ } 


\date{August 16, 2021}


\maketitle 

\begin{ques}
\begin{enumerate}
    \item A compound statement S is called a \textbf{contradiction} if it is false for all possible combinations of truth values of the component statements that are used to form S. For statements P and
Q, show that $(P \land (\neg Q)) \land (P \land Q)$ is a contradiction.
 \item A compound statement S is called a \textbf{tautology} if it is true for all possible combinations of
truth values of the component statements that comprise S. For statements P and Q, show
that $P \Rightarrow (P \lor Q)$ is a tautology.

\end{enumerate}
\end{ques}
\newpage

\begin{ques}
Prove that the following formulas hold for all $n \in N$:
\begin{enumerate}
    \item $\sum_{k=1}^n\frac{a-1}{a^k}=1-\frac{1}{a^n}$, where $a\neq 0.$
    \item $\sum_{k=1}^n(2k-1)^2=\frac{n(4n-1)^2}{3}$.
\end{enumerate}
\end{ques}

\newpage
\begin{ques}
 Prove the following properties of the set operations:
    \begin{enumerate}
        \item $(A\cap B)^c=A^c\cup B^c$;
         \item $(A\cup B)^c=A^c\cap B^c$;
          \item $A\cap B=(A^c\cup B^c)^c$;
          \item $A\cup B=(A^c\cap B^c)^c$.
    \end{enumerate}
\end{ques}


\newpage
\begin{ques}
A supervisor in a manufacturing plant has $m$ men and $w$ women working
for him. He wants to choose $k$ workers for a special job. Not wishing to show
any biases in his selection, he decides to select the  workers at random. Let
Y denote the number of women in his selection. 
\textit{Hint: } In general case, $p(y)=\frac{\begin{pmatrix} w\\ y \end{pmatrix}\begin{pmatrix}m \\k-y \end{pmatrix}}{\begin{pmatrix} w+m\\ k\end{pmatrix}}$ for $y=0,1,2,\cdots, k$.
\begin{enumerate}
    \item Find the probability
distribution for Y when there are 3 men and 3 women, and k=2.
\item Find $E[Y]$ and $VAR[Y]$.
\end{enumerate}
\end{ques}


\newpage 
\begin{ques}
If $Y$ is a continuous random variable with cdf $F(y)$ and pdf $f(y)$, then for any $a < b$, we have
\[P(a<Y<b)=P(a\leq Y< b) =P(a< Y\geq  b)=P(a\leq Y\geq b)
\]
\[=F(b)-F(a)=\int_a^b f(y)dy
\]
Suppose $Y$ is a continuous random variable with cdf
\[F(y)=\frac{1}{1+e^{-y}}, \text{ for }-\infty <y <\infty.
\]
Find the pdf of Y . Calculate $P(2 < Y < 2)$ using the cdf and the pdf.
\end{ques}


\newpage
\begin{ques}
If Y has density function

\[f(y)=\begin{cases}\frac{3}{2}y^2+y, & 0\leq y \leq 1,\\
0, & \text{everywhere else.}
\end{cases}
\]
Find the mean and the variance of $Y$.
\end{ques}

\newpage
\begin{ques}
Compute the rank of each of the following matrices:
\begin{enumerate}
    \item \[\begin{pmatrix} 2 & -4 & 2\\-1& 2& 1\end{pmatrix}
    \]
    \item \[\begin{pmatrix} 1 & 6 & -7 & 3\\ 1& 9 & -6 &4
    \\ 1& 3& -8 & 4\end{pmatrix}
    \]
    \item \[\begin{pmatrix} 1 & 6 & -7 & 3 &5\\ 1& 9 & -6 &4&9
    \\ 1& 3& -8 & 4& 2\\ 2&15&-13 &11 & 16 \end{pmatrix}
    \]
\end{enumerate}
\end{ques}

\newpage
\begin{ques}
Suppose 
\begin{align*}
    \begin{array}{cc}
     A=\begin{pmatrix} 2 & 3& 1\\ 0 & -1 & 2 
         \end{pmatrix},    &  B=\begin{pmatrix} 2& 1\\ 1& 1\end{pmatrix}.
    \end{array}
\end{align*}
Verify that $(BA)^T=A^TB^T$.
\end{ques}


\newpage
\begin{ques}
Solve the following systems of equations:
\begin{enumerate}
    \item \begin{align*}
    2x_1+x_2=5\\ x_1+x_2=3
\end{align*}
\item \begin{align*}
    2x_1+x_2=4\\
    6x_1+2x_2+6x_3=20\\
    -4x_1-3x_2+9x_3=3;
\end{align*}
\item \begin{align*}
    2x_1+4x_2=2\\
    4x_1+6x_2+3x_3=1\\
    -6x_1-10x_2=-6.
\end{align*}
\end{enumerate}
\end{ques}
\newpage
\begin{ques}
Find the eigenvalues and eigenvectors of the following matrices
\begin{align*}
\begin{array}{cc}
  A=  \begin{pmatrix}
    -1& 3\\
    2&0
    \end{pmatrix},
&  B=\begin{pmatrix} 
1 & 0& 2\\
0& 5& 0\\
3 &0 &2
\end{pmatrix}.
\end{array}
\end{align*}
\end{ques}

\newpage
\begin{ques}
Determine the definiteness of the following matrices:
\begin{align*}
    \begin{array}{ccc}
        \begin{pmatrix}
         2 &-1\\ -1 & 1
        \end{pmatrix}, & \begin{pmatrix}
         -3 & 4\\ 4 &-5
        \end{pmatrix}, &
        \begin{pmatrix}
         2 &4\\ 4 &8
        \end{pmatrix}
         \\ \begin{pmatrix}
         1&2 &0\\ 2& 4 & 5\\
         0 & 5&6
        \end{pmatrix}&
        \begin{pmatrix}
         -1& 1& 0\\ 1&-1 &0 \\ 0&0&-2
        \end{pmatrix}, & \begin{pmatrix}
         1&0&3&0\\ 0&2&0&5\\ 3& 0& 4&0\\ 0&5&0&6
        \end{pmatrix}.
    \end{array}
\end{align*}
\end{ques}


\end{document}
