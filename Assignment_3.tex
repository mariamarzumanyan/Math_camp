
\documentclass[12pt]{article}
\usepackage{amsmath}
\usepackage{graphicx,epsfig, wrapfig}
\usepackage{amssymb}
\usepackage{setspace}
\usepackage{amsthm}
\usepackage{subfig}
\usepackage{amsmath}
%\usepackage{chicago}
\usepackage{natbib}
\usepackage{enumerate}
\usepackage{url}
\usepackage{epstopdf}
\usepackage{fancyhdr}
\usepackage{threeparttable}
\usepackage{caption}
\usepackage{booktabs}
\usepackage{float}
\usepackage{array}
\usepackage{multirow}
%\usepackage[T1]{fontenc}
%\usepackage[utf8]{inputenc}
%\usepackage{authblk}
\newcommand{\keyword}[1]{\textsf{\slshape #1}}

 
\setcounter{MaxMatrixCols}{10}    

%\setstretch{3}
\onehalfspacing
%\setlength{\footnotesep}{0.5cm}
%\doublespacing
\newtheorem{theorem}{Theorem}   
\newtheorem{acknowledgment}[theorem]{Acknowledgment}   
\newtheorem{algorithm}[theorem]{Algorithm}
\newtheorem{axiom}[theorem]{Axiom}
\newtheorem{case}[theorem]{Case} 
\newtheorem{claim}[theorem]{Claim} 
\newtheorem{conclusion}[theorem]{Conclusion}
\newtheorem{condition}[theorem]{Condition}
\newtheorem{conjecture}[theorem]{Conjecture}
\newtheorem{corollary}[theorem]{Corollary}
\newtheorem{criterion}[theorem]{Criterion}
\newtheorem{definition}[theorem]{Definition}
\newtheorem{example}[theorem]{Example}
\newtheorem{exercise}[theorem]{Exercise}
\newtheorem{lemma}{Lemma}
\newtheorem{notation}[theorem]{Notation}
\newtheorem{problem}[theorem]{Problem}
\newtheorem{ques}[theorem]{Question}
\newtheorem{proposition}{Proposition}
\newtheorem{remark}[theorem]{Remark}
\newtheorem{solution}[theorem]{Solution}
\newtheorem{summary}[theorem]{Summary} 
\setlength{\oddsidemargin}{0mm}
\setlength{\textwidth}{6.5in}
\setlength{\topmargin}{0in}
\setlength{\headheight}{0in}
\setlength{\headsep}{0in}
\setlength{\textheight}{9in}


\newtheorem{prop}{Proposition}
 

\makeatletter
\newcommand*{\rom}[1]{\expandafter\@slowromancap\romannumeral #1@}
\makeatother

\newcounter{saveenumi}
\newcommand{\seti}{\setcounter{saveenumi}{\value{enumi}}}
\newcommand{\conti}{\setcounter{enumi}{\value{saveenumi}}}

\begin{document}


\setcounter{footnote}{0}
\title{Mathematics for Economics PhD: Assignment \rom{3}}
\author{\textsc{Mariam Arzumanyan,}\thanks {   E-mail: mariama2@illinois.edu.} \\ } 


\date{August 18, 2021}


\maketitle 


\begin{ques}
Which of the following functions are homogeneous? What are the degrees of homogeneity of the homogeneous ones?
\begin{enumerate}
    \item $3x^5y+2x^2y^4-3x^3y^3$,
    \item $3x^5y+2x^2y^4-3x^3y^4$,
    \item $x^{\frac{1}{2}}y^{-\frac{1}{2}}+3xy^{-1}+7$,
    \item $x^{\frac{3}{4}}y^{\frac{1}{4}}+6x$,
    \item $x^{\frac{3}{4}}y^{\frac{1}{4}}+6x+4$,
    \item $\frac{x^2-y^2}{x^2+y^2}+3$.
\end{enumerate}
\end{ques}

\newpage
\begin{ques}
Consider the constant elasticity of substitution (CES) production function
\[F(x_1, x_2)=A(a_1x_1^\rho+a_2x_2^\rho)^\frac{1}{\rho}.
\]
Show that $F$ has constant returns to scale. 
\end{ques}

\newpage
\begin{ques}
Which of the following functions are homothetic?
\begin{enumerate}
    \item $e^{x^2y}e^{xy^2}$,
    \item $4 \log x +3\log y$,
    \item $x^3y^6+3x^2y^4+6xy^2+9$,
    \item $x^2y+xy$,
    \item $\frac{x^2y^2}{xy+1}$.
   % \item 
\end{enumerate}
\end{ques}



\newpage
\begin{ques}
Consider the general Cobb-Douglas function on $\mathbb{R}_+^2$: 
\[u(x,y)=x^ay^b, a>0, b>0.
\]
Is it a convex function? Is it a concave function? Is  it quasiconcave/quasiconvex?

\end{ques}


\newpage
\begin{ques}
Which of the following functions on $\mathbb{R}^n$ are concave or convex?
\begin{enumerate}
    \item $f(x)=3e^x+5x^4-\ln x$,
    \item $f(x,y)=-3x^2+2xy-y^2+3x-4y+1$,
    \item $f(x,y,z)=3e^x+5y^4-\ln z$,
    \item $f(x,y,z)=Ax^ay^bz^c$, for $a,b,c>0$.
\end{enumerate}
\end{ques}


\newpage
\begin{ques}
Which of the following functions on $\mathbb{R}$ are quasiconcave or quasiconvex?
\begin{enumerate}
    \item $f(x)=e^x$,
    \item $f(x)=\ln x$,
    \item $f(x)=x^3+x$,
    \item $f(x)=x^3-x$,
    \item $f(x)=x^4-x^2$,
    \item $f(x)=x^4+x^2$,
    \item $f(x)=3x^3-5x^2+7x$,
    \item $f(x)=\sin x$.
\end{enumerate}

\end{ques}

\newpage
\begin{ques}
Which of the following functions on $\mathbb{R}^n$ are quasiconcave or quasiconvex?
\begin{enumerate}
    \item $f(x,y)=ye^{-x}$ on $\mathbb{R}_+^2$.
    \item $f(x,y)=(2x+3y)^3$ on $\mathbb{R}^2$.
    \item $f(x,y)=\frac{y}{x^2+1}$ on $\mathbb{R}_+^2$.
    \item $f(x,y)=yx^{-2}$ on $\mathbb{R}_+^2$.
\end{enumerate}
\end{ques}


\newpage
\begin{ques}
Find the local maxs and mins of $f(x,y)=x^3-y^3+9xy$.
\end{ques}

\newpage
\begin{ques}
For each of the following functions defined on $\mathbb{R}^2$, find the critical points and classify these as local max, local min, saddle point, global max, global min, or "can't tell".
\begin{enumerate}
    \item $x^4+x^2-6xy+3y^2$,
    \item $x^2-6xy+2y^2+10x+2y-5$,
    \item $xy^2+x^3y-xy$,
    \item $3x^4+3x^2y-y^3$.
    
\end{enumerate}
\end{ques}


\newpage
\begin{ques}
Find the maximum and minimum distance from the origin to the ellipse $x^2+xy+y^2=3$. [Hint: Use $x^2+y^2$ as your objective function.]
\end{ques}
\newpage
\begin{ques}
Solve a utility maximization problem on $\mathbb{R}_+^2$:
\[\max_{x,y} u(x,y)=xy
\]
subject to 
\[2x+3y=18.\]
\[x\geq 0, y\geq 0
\]
\end{ques}

\newpage

\begin{ques}
Solve a utility maximization problem on $\mathbb{R}\times (0,\infty)$:
\[\max_{x,y} u(x,y)=x+\ln y.
\]
subject to 
\[p_xx+p_yy\leq w.\]
\end{ques}

\newpage
\begin{ques}
Find the general expression for the commodity bundle $(x_1, x_2)$ which maximizes the Cobb-Douglas utility function $u(x,y)=x^\alpha y^\beta$ on the budget set $p_xx+p_yy\leq w$, $x_1\geq 0$, $x_2\geq 0$.
\end{ques}



\newpage
\begin{ques}
Solve a utility maximization problem on $\mathbb{R}_+^2$:
\[\max_{x,y} u(x,y)=\min \{x,y\}.
\]
subject to 
\[p_xx+p_yy\leq w.\]
\end{ques}
\newpage
\begin{ques}
Solve a utility maximization problem on $\mathbb{R}_+^2$:
\[\max_{x,y} u(x,y)=\max \{x,y\}.
\]
subject to 
\[p_xx+p_yy\leq w.\]
\end{ques}
\newpage
\begin{ques}
Solve a utility maximization problem on $\mathbb{R}_+^2$:
\[\max_{x,y} u(x,y)=x+y.
\]
subject to 
\[p_xx+p_yy\leq w.\]
\end{ques}



\newpage
\begin{ques}
Example 18.5 Textbook.
\[\max_{x,y} f(x,y)=x^2y
\]
subject to $$2x^2+y^2=3$$.
\end{ques}
\end{document}
